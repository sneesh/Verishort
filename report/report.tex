\documentclass[letterpaper,12pt]{article}
\usepackage[letterpaper,margin=1in]{geometry}
%\setlength{\topmargin}{-.5in}
%\setlength{\textheight}{9in}
%\setlength{\oddsidemargin}{.125in}
%\setlength{\textwidth}{6.25in}

\begin{document}
\title{Verishort HDL}
\author{Anish Bramhandkar \ Elba Garza \ Scott Rogowski \ Ruijie Song}
\renewcommand{\today}{December 22, 2010}
\maketitle

\newpage

\tableofcontents

\newpage

\section{An Introduction to Verishort}
    \subsection{Background}
    \indent Verilog is a very popular hardware description language (HDL) which is widely utilized by
    the electronics hardware design industry. First invented and used in the early 80s at
    Automated Integrated design Systems, Verilog was put into the public domain and
    standardized by the IEEE in 1995. This initial public version of Verilog became known as 
    Verilog-95. The language was later expanded in 2001 and 2005 to address deficiencies
    and add features resulting in Verilog-2001 and Verilog-2005, the most recent version. (A
    combined hardware description\slash verification language known as SystemVerilog was 
    extended from the 2005 standard but goes beyond the scope of this manual.) \\
    \indent Despite its popularity, Verilog is infamous for its repetitiveness, strange grammar, and 
    ease of bug insertion. Part of this is a factor of the nature of low-level hardware design. 
    There is a difference between languages meant to be run using gates and latches rather 
    than processors and memory. However, we believe that another part of this simply poor 
    language design and can be improved. \\
    \indent VeriShort HDL is meant to simplify the Verilog-2005 language to make it easier to read 
    and write. First, we have reduced repetitiveness in accordance with the DRY (Don�t 
    repeat yourself) philosophy by simplifying module input/output syntax and instantiation. 
    Next, we introduced some C-language features such as brackets and array-like bus 
    descriptions. We substantially simplified synchronous logic by doing away with �always� 
    syntax and replacing it with simple �if� statements. The list of reserved keywords has been 
    substantially shortened in order to make VeriShort completely synthesizable and to 
    remove rarely used features. Finally, we added a standard library of commonly used 
    electronic components like latches, multiplexers, and decoders to further reduce the 
    Verilog tedium. \\
    \indent Because of the wide adoption of Verilog and the existence of many verifiers and 
    hardware synthesizers specific to the IEEE standards, the initial goal of VeriShort will not 
    be to exist as a self contained HDL but rather to translate into clean synthesizable Verilog 
    code. In support of these efforts, a translator has been started and is expected to be 
    running by the date of December 22nd 2010.

    \subsection{Related Work}
        \subsection{Goals of Verishort}
        \subsubsection{Short}
        \subsubsection{Logical}
        \subsubsection{Clean}


\section{Tutorial}
    \subsection{A First Example}
    \subsection{Compiling and Running a Verishort file}
    \subsection{More Examples}

\section{Reference Manual}
    \subsection{Lexical Conventions}
        \subsubsection{Tokens}
        There are 5 classes of tokens: identifiers, keywords, numbers, operators, and other separators.
        Blanks, tabs, and newlines (collectively, whitespace) are ignored, except when they serve
        to separate tokens.
        \subsubsection{Comments}
        The characters \slash* introduce a comment, which is terminated by the characters *\slash. The characters 
        \slash\slash also introduce a comment, which is terminated by the newline character. Comments do not 
        nest. Lines marked as comments are discarded by the compiler.
        \subsubsection{Data Types}
        The primary data type is the bit, which may store the value 0 or 1. A group of bits comprises a bus. All 
        multibit binary values are treated as two�s complement numbers.
        \indent In for-loops (see the corresponding section), the loop variable is assumed to be a simple integer (i.e., 
        natural number).
        \subsubsection{Identifiers}
        An identifier is a sequence of characters that represent a wire, bus, register, parameter, or module. An 
        identifier may only include alphanumerical characters or the underscore character (\_). The first character 
        of an identifier may not be a number.
        \subsubsection{Keywords}
        The following identifiers are reserved as keywords and may not be used for any other purpose: 
        In this manual, keywords are bolded.
        \begin{itemize}
        \item{\textbf{case}}
        \item{\textbf{clock}} 
        \item{\textbf{else}}
        \item{\textbf{for}}
        \item{\textbf{if}}
        \item{\textbf{input}}
        \item{\textbf{module}}
        \item{\textbf{negedge}}
        \item{\textbf{output}}
        \item{\textbf{parameter}}
        \item{\textbf{posedge}}
        \item{\textbf{register}}
        \item{\textbf{return}}
        \item{\textbf{reset}}
        \item{\textbf{wire}}
        \end{itemize}
        \subsubsection{Numbers}
        Numbers can be either binary or integer values and are specified as follows: 
        \begin{itemize}
        \item{A sequence of digits, followed by a radix suffix (b for binary, or d for integer)}
        \item{If there is only a single 0 or 1, no suffix needs to be provided because 0 and 1 are
        equivalent in integer and binary} 
        \item{The characters 0 and 1 are valid binary digits.} 
        \item{The characters 0-9 are valid integer digits.}
        \item{Extended binary numbers are like normal binary numbers, but may also use the
        character �x� as a binary digit. They may only be used in case structures.}
        \end{itemize}
        \subsubsection{Operators}
        An operator is a token that specifies an operation on at least one operand. The operand 
        may be an expression or a constant. \\
        Bitwise operators:
        \begin{itemize}
        \item{!} 
        \item{\& }
        \item{ !\& } 
        \item{$\mid$}
        \item{ !$\mid$} 
        \item{\textasciicircum }
        \item{ !\textasciicircum} 
        \item{ \textless\textless} 
        \item{ \textgreater\textgreater}
        \end{itemize}
        Comparison operators: 
        \begin{itemize}
        \item{ ==}
        \item{ \textgreater= }
        \item{ \textless= }
        \item{\textgreater }
        \item{\textless}
        \end{itemize}
        Arithmetic operators: 
        \begin{itemize}
        \item{+}
        \item{- }
        \item{* }
        \item{/ }
        \item{\%} 
        \end{itemize}
        Assignment operator: 
        \begin{itemize}
        \item{=}
        \end{itemize}
        Sign extension operator: 
        \begin{itemize}
        \item{ ' }
        \end{itemize}
        \subsubsection{Buses}
        \indent A bus represents a multibit wire. The number of bits in a bus must be determinable at compile 
        time. Buses are declared using the syntax data\_type bus\_name[number\_of\_bits]; Where 
        data\_type is either wire, register or assumed to be input or output by its position in a module 
        declaration. The number of bits must be a constant. From here, any bit in the bus may be referred 
        to using the subscript syntax: bus\_name[bit\_index], where bit\_index is a constant or expression 
        that yields an integer value less than or equal to the size of the bus. \\
        \indent A range of bits in a bus is represented by using the index of the most significant bit in the 
        range, followed by the colon character (:), followed by the index of the least significant 
        bit in the range, as the subscript. E.g. wires 4-8 would be referred to by bus\_name[7:3]. 
        Reversing this order is invalid.
        
    \subsection{Assignment}
    All assignments in Verishort bind wires and ports to other wires, binary values, or decimal values. 
    These can be done en masse, such as in buses (multi-bit wires) or one by one.
    
        \subsubsection{Assigning Wires}
        
        The most basic assignment is of a single wire to a single bit value: \\\\
        \textbf{wire} w1 = 0;  \\
        \textbf{wire} w2 = !w1;     \slash\slash w2 == 1\\\\
        A bundle of wires can be assigned to a multi-bit value, as long as the number of bits matches 
        the number of wires in the bundle: \\\\
        \textbf{wire} w3[5] = 01010b;	\slash\slash assigning a binary value \\
        \textbf{wire} w4[4] = 10d;		\slash\slash assigning a decimal value \\
        \textbf{wire} w5[10] = {1,8{0},1};	\slash\slash 1000000001b using concatenation \\\\
        \slash* This does not work and will result in an error because the left hand side and the right hand 
        side are not the same size. *\slash \\
        \textbf{wire} w6[10] = 10b; 
        

\section{Project Plan}
    \subsection{Team Responsibilities}
    \subsection{Project Timeline}
    \subsection{Software Development Environment}
    \subsection{Project Log}
    
\section{Architectural Design}
    \subsection{Architecture}
    \subsection{the Runtime Environment}
    \subsection{Error Recovery}

\section{Testing Plan}
    \subsection{Goals}
    \subsection{Hypothesis}
    \subsection{Methods}
        \subsubsection{Phase I}
        \subsubsection{Phase II}
        \subsubsection{Phase III}
    \subsection{Tools}
    \subsection{Implementation}
        \subsubsection{Phase I}
        \subsubsection{Phase II}
        \subsubsection{Phase III}
        
\end{document}
